\documentclass{article}
\usepackage{amsmath,amssymb}
\title{JPEG 2000 \& Wavelet Compression}
\author{Ryan Brown}

\begin{document}
	\maketitle
	JPEG 2000 Goals: t was intended to be scalable in nature, allowing people to obtain a representation of the image at lower resolution from the coldstream (data after compression).  This makes the compression more complex and computationally demanding, but increases performance in the decompression. \\ 

	Cons:  produces ringing artifcats near the edges of the images, or tiles.

	Steps: 

	1) Color Component Transform \\
	2) Tiling (not necessary, can just lessen memory needed)\\ 
	3) Wavelet Transformation (name the two types then general info below) \\ 
	4) Quantization \\ 


	Wavelet Transform
	\begin{align*}
		Wf(u,\lambda) := \left< f, \psi_{u,\lambda} \right> \\ = \lambda^{-\frac{1}{2}}\int_{\mathbb{R}} f(t) \psi \left( \lambda^{-1}(t-u)\right) dt 
	\end{align*}
	\begin{align*}
		\widetilde{\psi_{\lambda}}(u) = \lambda^{-\frac{1}{2}} \overline{\psi(\lambda^{-1}t)} \\ 
		Wf(u,\lambda) := f * \widetilde{\psi_{\lambda}}(u)
	\end{align*}


	\begin{align*}
		\psi_j [n] = a^{-\frac{a^j}{2}} \psi \left(\frac{n}{a^j}\right)
	\end{align*}

	Treat $f[n]$ and $\psi_j [n]$ as periodic signals of period N.  Then we write with circular convultion with $\overline{\psi_j}[n] = \psi_j^*[-n]$
	\begin{align*}
		Wf[n,a^j] = \sum_{m=0}^{N-1} f[m]\psi_j^* [m-n] 
	\end{align*}
	We only need to store frequencies at a level equal to twice the highest frequency (maybe a bit higher).  

	We'll end up using a dyadic wavelet transform with $\lambda = 2^j$
	We can then bound the sum of $Wf$ with constants. (page 209)

	A discrete dyadic wavelt transform can be computed with a fast filter bank algorithim if the wavelet is designed properly, similar to how we did it in class. (page 212)




	Filter Bank Calculation
	(Talk about how to choose filters)

	\begin{align*}
		T(N) = 2N + T\left(\frac{N}{2}\right) \\ 
		=  2N + 1N [ T\left(\frac{N}{4}\right) + N] \\ 
		= 3N + T\left(\frac{N}{4}\right) \\ 
		= 3N + [\frac{N}{2} + T\left(\frac{N}{8}\right) ]
	\end{align*}
	\begin{align*}
		2+ \sum_{i=0}^{\log N} (2)^{-i} \\ 
		2 + \frac{1-(\frac{1}{2})^{\log N}}{1-\frac{1}{2}} \\
		= 2 + 2 - \frac{2}{N} \\ 
		= 4 - \frac{2}{N}  
	\end{align*}
	\begin{align*}
		O(T(N)) = 4N -2 = O(N)
	\end{align*}

	With the locality of wavelets, we can calculate a bit at a time instead of needing the full signal. \\ 

	Lifting to improve time (not complexity but the multiplier) Unsure if I want to touch on this, I will for paper\\

	4) Quantization (Need to finish this part)


\end{document}	